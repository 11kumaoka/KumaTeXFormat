\begin{center}
	\textbf{概要}
\end{center}

To evaluate the physical property of the quark-gluon plasma (QGP),
we use jets observed in relativistic heavy-ion collisions experiments.
Jet's properties such as yield, transverse momentum, shape, and etc,
are modified by interaction between the QGP and the jet's source particle.
Estimation of this modification will clarify the stopping power of the QGP.
As an indication of the stopping power, 
we calculate the nuclear modification factor ($\raa$) defined 
as the ratio of the jet yield in Pb–Pb collisions to the scaled jet yield in $pp$ collisions.

In the preceding study\cite{Osada:2019oor} by using data 
taken in Pb-Pb collisions at $\cmeNN = 5.02~\TeV$ with LHC-ALICE in 2015,
the $\raa$ is lower than 1 and it means the existence of the jet suppression effect.
On the other hand, these results have only the limited $\pt$ range and centrality 
caused by small statistics and large background.

This study will evaluate the $\raa$ in the higher $\pt$ region 
by using new large statistic data, and in the lower $\pt$ region 
by using peripheral collision data with less background.
Furthermore, the study using various centrarity data 
will clarify the centrality dependence of the $\raa$.
